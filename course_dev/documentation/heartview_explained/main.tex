%%%%%%%%%%%%%%%%%%%%%%%%%%%%%%%%%%%%%%%%%
% The Legrand Orange Book
% LaTeX Template
% Version 2.4 (26/09/2018)
%
% SOURCE: https://www.latextemplates.com/template/the-legrand-orange-book
%
% This template was downloaded from:
% http://www.LaTeXTemplates.com
%
% Original author:
% Mathias Legrand (legrand.mathias@gmail.com) with modifications by:
% Vel (vel@latextemplates.com)
%
% License:
% CC BY-NC-SA 3.0 (http://creativecommons.org/licenses/by-nc-sa/3.0/)
%
% Compiling this template:
% This template uses biber for its bibliography and makeindex for its index.
% When you first open the template, compile it from the command line with the 
% commands below to make sure your LaTeX distribution is configured correctly:
%
% 1) pdflatex main
% 2) makeindex main.idx -s StyleInd.ist
% 3) biber main
% 4) pdflatex main x 2
%
% After this, when you wish to update the bibliography/index use the appropriate
% command above and make sure to compile with pdflatex several times 
% afterwards to propagate your changes to the document.
%
% This template also uses a number of packages which may need to be
% updated to the newest versions for the template to compile. It is strongly
% recommended you update your LaTeX distribution if you have any
% compilation errors.
%
% Important note:
% Chapter heading images should have a 2:1 width:height ratio,
% e.g. 920px width and 460px height.
%
%%%%%%%%%%%%%%%%%%%%%%%%%%%%%%%%%%%%%%%%%

%----------------------------------------------------------------------------------------
%	PACKAGES AND OTHER DOCUMENT CONFIGURATIONS
%----------------------------------------------------------------------------------------

\documentclass[11pt,fleqn]{book} % Default font size and left-justified equations

\input{structure.tex} % Insert the commands.tex file which contains the majority of the structure behind the template

%\hypersetup{pdftitle={Title},pdfauthor={Author}} % Uncomment and fill out to include PDF metadata for the author and title of the book

\usepackage[printonlyused,withpage]{acronym} 
\usepackage{xcolor}

%----------------------------------------------------------------------------------------

\begin{document}

%----------------------------------------------------------------------------------------
%	TITLE PAGE
%----------------------------------------------------------------------------------------

\begingroup
\thispagestyle{empty} % Suppress headers and footers on the title page
\begin{tikzpicture}[remember picture,overlay]
\node[inner sep=0pt] (background) at (current page.center) {\includegraphics[width=\paperwidth]{background.pdf}};
\draw (current page.center) node [fill=ocre!30!white,fill opacity=0.6,text opacity=1,inner sep=1cm]{\Huge\centering\bfseries\sffamily\parbox[c][][t]{\paperwidth}{\centering HeartView\\[15pt] % Book title
{\Large A Remote Testing Station for Pacemakers}\\[20pt] % Subtitle
{\huge Guy Meyer}}}; % Author name
\end{tikzpicture}
\vfill
\endgroup

%----------------------------------------------------------------------------------------
%	COPYRIGHT PAGE
%----------------------------------------------------------------------------------------

\newpage
~\vfill
\thispagestyle{empty}

\noindent TODO: GPLv3 Copyright \copyright\ 2019 John Smith\\ % Copyright notice

\noindent \textsc{Published by Publisher}\\ % Publisher

\noindent \textsc{book-website.com}\\ % URL

\noindent Licensed under the Creative Commons Attribution-NonCommercial 3.0 Unported License (the ``License''). You may not use this file except in compliance with the License. You may obtain a copy of the License at \url{http://creativecommons.org/licenses/by-nc/3.0}. Unless required by applicable law or agreed to in writing, software distributed under the License is distributed on an \textsc{``as is'' basis, without warranties or conditions of any kind}, either express or implied. See the License for the specific language governing permissions and limitations under the License.\\ % License information, replace this with your own license (if any)

\noindent \textit{First printing, March 2019} % Printing/edition date

%----------------------------------------------------------------------------------------
%	TABLE OF CONTENTS
%----------------------------------------------------------------------------------------

%\usechapterimagefalse % If you don't want to include a chapter image, use this to toggle images off - it can be enabled later with \usechapterimagetrue

\chapterimage{chapter_head_1.pdf} % Table of contents heading image

\pagestyle{empty} % Disable headers and footers for the following pages

\tableofcontents % Print the table of contents itself

\cleardoublepage % Forces the first chapter to start on an odd page so it's on the right side of the book

\pagestyle{fancy} % Enable headers and footers again

\section*{List of Acronyms}
\addcontentsline{toc}{chapter}{List of Acronyms}
\begin{acronym}
	%Define acronyms here
	%Use intext with \ac{}. Look at reference for the acronym package for a full guide
	\acro{ADC}{analog to digital convertor}
	\acro{MCU}{microcontroller}
	\acro{PCB}{printed circuit board}
	\acro{OS}{operating system}
	\acro{UART}{universal asynchronous receiver-transmitter}

\end{acronym}

%----------------------------------------------------------------------------------------
%	PART
%----------------------------------------------------------------------------------------

%\part{Part One}

%----------------------------------------------------------------------------------------
%	CHAPTER 1
%----------------------------------------------------------------------------------------

\chapterimage{chapter_head_2.pdf} % Chapter heading image

\chapter{Introduction}

\section{What is it?}\index{Description}

Learn about pacemakers! HeartView is an all-in-one remote testing system for designing pacemaker prototypes. Users can create models using their favourite development environment, deploy to the device, and test in a real-time environment. 

%------------------------------------------------


\section{The Pacemaker Challenge}\index{Pacemaker Challenge}

Boston Scientific has released into the public domain the system specification for a previous generation pacemaker. A major reason for publishing this specification is to have it serve as the basis for a challenge to the formal methods community, in the spirit of other Grand Challenges \cite{sqrl_pacemaker}.


%------------------------------------------------

\section{Motivation}\index{Motivation}

Over the last few years our group at McMaster University have converted this challenge into an integrated lab supporting a course on software development. The document supports the design process for the students and the development of the pacemaker is split into several assignments completed over the course of the semester. This hands-on learning style has shown success in recent years, seen through student engagement, attendance, and enrollment.

The evolution of this product jumped when schools closed due to the global pandemic sending students to learn online. The need for a remote method of development gave rise to HeartView, a remote testing system for pacemaker. Students can now study the mechanics of a pacemaker, and with the physiology of the heart anywhere they go.


%------------------------------------------------


%----------------------------------------------------------------------------------------
%	CHAPTER 2
%----------------------------------------------------------------------------------------

\chapter{The HeartView System}

\section{System Components}\index{System Components}

The HeartView testing environment is composed of three pillars software, electrical, and hardware. 

%------------------------------------------------

\subsection{Software}\index{System Components!Software}

The software used for this project can be broken down into two components; the tool used to control the physiology of the heart (used for testing), and that used to program the pacemaker.

%------------------------------------------------

\subsubsection{Heart}\index{System Components!Software!Heart}

The software used to control the heart is the HeartView UI. Meant to be installed on your local machine and communicate with the testing system via serial communication. The HeartView UI receives data from the heart, dispatches test routines, plots real-time data, and supports data analysis, as well as report generation.

%------------------------------------------------

\subsubsection{Pacemaker}\index{System Components!Software!Pacemaker}

In the first version of HeartView (that used during the creation of this document), the \ac{MCU} used for the pacemaker is the FRDM K64F. Software can be written using Mbed Studio, MATLAB Simulink (used at McMaster), or other supported frameworks. The essence is to generate a binary file (.bin) that can be flashed (uploaded) onto the board. Learn more below \nameref{Development and Testing!Programming a Pacemaker}.

%------------------------------------------------


\subsection{Electrical}\index{System Components!Electrical}

In order to create a cardiac simulator two components have to work in parallel, the heart and the pacemaker. 

%------------------------------------------------

\subsubsection{Heart}\index{System Components!Electrical!Heart}

The \ac{MCU} used to power the heart heart is the Nucleo F446RE. It comes equipped with a \ac{PCB} shield that shares signals with the pacemaker via ribbon cables (mimicking the functionality of the electrodes). As an overview, the Nucleo collects electrical signals from the pacemaker using an AD8220 Instrumentation Amplifier and an onboard \ac{ADC}. Signals are sampled and sent to the HeartView UI for plotting via \ac{UART}, ie. serial communication. The Nucleo is also responsible for generating natural heart signals according to the test routine constructed in the HeartView UI. Every time a new test routine is dispatched from HeartView, the test is parsed and implmented by the Nucleo in real-time. This is what provides the interactive experience of the HeartView Testing Station. The Nucleo shield also includes a pair of LEDs that pulse with every natural pace giving the user visual indication of natural heart activity.

%------------------------------------------------
\subsubsection{Pacemaker}\index{System Components!Electrical!Pacemaker}

The pacemaker is a FRDM K64F \ac{MCU} with a shield designed to mimic the electrical composition of a the real implantable device. More can be learned about the pacemaker shield in the ``Pacemaker Shield Explained" document.

%------------------------------------------------

\subsection{Hardware}\index{System Components!Hardware}

Though most of the complexity is in the software and electrical components, the hardware was not trivial. In essence, the hardware is meant to encase the test station and create a comfortable testing environment. Since it is unclear what space is available for the students, priority was given to creating a small form factor device that can protect the electronics without restricting access to key peripherals like the user buttons, and serial connectors. Laser cut covers were made for both sides to ensure that the students can study the electronics without the risk of damaging the device.

\begin{figure}[h]
	\centering\includegraphics[width=\textwidth]{heartview_overview.png}
	\caption{HeartView testing station hardware}
	\label{fig:heartview_overview} % Unique label used for referencing the figure in-text
	%\addcontentsline{toc}{figure}{Figure \ref{fig:placeholder}} % Uncomment to add the figure to the table of contents
\end{figure}


%------------------------------------------------

\section{Downloading the Software}\index{Downloading the Software}\label{sec:downloading_the_software}

Since HeartView is licensed under the GPLv3 Open Source License, the software is available for free on Github. Follow the link corresponding to your \ac{OS} where you can find a zip file that includes the bundled executable.

\subsection{Windows}\index{Installation!Windows}

Windows 10 >>\\ \url{https://github.com/theguymeyer/heartview/blob/master/course_dev/ui-HeartView/win/exe/heartview_win10.zip}

\subsection{MacOS}\index{Installation!MacOS}

MacOS 10.4 or newer >>\\ \url{https://github.com/theguymeyer/heartview/blob/master/course_dev/ui-HeartView/mac/exe/heartview_mac.zip}

\subsection{Linux (Debian/Ubuntu)}\index{Installation!Linux}

Not Supported (Coming Soon!)

%------------------------------------------------

\section{Setting It Up}\index{Setting It Up}

To get started with the test station follow the following setups:

\begin{enumerate}
	\item Download the .zip (see Sec \ref{sec:downloading_the_software})
	\item Unzip
	\item Run the HeartView executeable (double click the icon)
		\begin{itemize}
			\item MacOS may list this app as an Unknown Developer. To overcome this right-click the icon and select ``Open" from the drop-down menu
		\end{itemize}
	\item Connect the 4 pin cables between the heart and pacemaker (as shown in Figure \ref{fig:heartview_cables})
	\item Plug in the test station using both USB cables provided
		\begin{itemize}
			\item \underline{FRDM K64F connects using a mini-USB}. Ensure that you are connected to the OpenSDA port (on right hand-side of the ethernet port when viewing from the top with the pacemaker on the left side of the testing station).
			\item \underline{Nucleo F446RE connects using a micro-USB}.
		\end{itemize}
	\item Press refresh on the serial options to see the connected devices
	\item Find the correct serial ID for the Heart (nucleo)
		\begin{itemize}
			\item On MacOS would be similar to `cu.usbmodem144203'
			\item On Windows would be similar to `COM7'
		\end{itemize}
	\item Press connect
	\item Setup the characteristics you want
	\item Dispatch the test routine by pressing the ``Dispatch" button
\end{enumerate}

\begin{figure}[h]
	\centering\includegraphics[width=\textwidth]{heartview_cables.png}
	\caption{HeartView testing station with connected cables}
	\label{fig:heartview_cables} % Unique label used for referencing the figure in-text
	%\addcontentsline{toc}{figure}{Figure \ref{fig:placeholder}} % Uncomment to add the figure to the table of contents
\end{figure}

%------------------------------------------------



\section{Development and Testing}\index{Development and Testing}

\subsection{Programming a Pacemaker}\index{Development and Testing!Programming a Pacemaker}

The pacemaker can be developed using your preferred environment where compatible binaries can be generated. At McMaster University, the MATLAB Simulink environment is used. The Simulink tool provides an easy-to-use interface with support for the FRDM K64F via the \href{https://www.mathworks.com/matlabcentral/fileexchange/55318-simulink-coder-support-package-for-nxp-frdm-k64f-board}{\textcolor{blue}{\underline{development support package}}}. Based on the pacing mode you'd like to develop, use the Simulink blocks, charts, and subsystems to model the correct functionality. The tool is also an effective way of abstracting the code and focusing of functional correctness. Once you are comfortable with the code and would like to begin real-time testing flash the board and use HeartView.

To learn about the pacemaker and to understand how to detect and generate pulses read the Pacemaker Shield Explained document. 

%------------------------------------------------

\subsection{Controlling the Heart}\index{Development and Testing!Controlling the Heart}

In this section we're going to dive into each components of the HeartView UI. 

%------------------------------------------------


\section{Generating Reports}\index{Generating Reports}


%------------------------------------------------

%----------------------------------------------------------------------------------------
%	PART
%----------------------------------------------------------------------------------------

%\part{Part Two}

%----------------------------------------------------------------------------------------
%	CHAPTER 3
%----------------------------------------------------------------------------------------

\chapterimage{chapter_head_1.pdf} % Chapter heading image

\chapter{Developer Notes}


\section{The Github Project}\index{github}


\section{HeartView UI - PyQt5}\index{pyqt5}


\section{The Heart - CubeMX}\index{cubemx}

Include the README?

%------------------------------------------------

%----------------------------------------------------------------------------------------
%	BIBLIOGRAPHY
%----------------------------------------------------------------------------------------

\chapter*{Bibliography}
\addcontentsline{toc}{chapter}{\textcolor{ocre}{Bibliography}} % Add a Bibliography heading to the table of contents

%------------------------------------------------

\section*{Articles}
\addcontentsline{toc}{section}{Articles}
\printbibliography[heading=bibempty,type=article]

%------------------------------------------------

\section*{Books}
\addcontentsline{toc}{section}{Books}
\printbibliography[heading=bibempty,type=book]


%------------------------------------------------

\section*{Online}
\addcontentsline{toc}{section}{Online}
\printbibliography[heading=bibempty,type=online]

%----------------------------------------------------------------------------------------
%	INDEX
%----------------------------------------------------------------------------------------

\cleardoublepage % Make sure the index starts on an odd (right side) page
\phantomsection
\setlength{\columnsep}{0.75cm} % Space between the 2 columns of the index
\addcontentsline{toc}{chapter}{\textcolor{ocre}{Index}} % Add an Index heading to the table of contents
\printindex % Output the index

%----------------------------------------------------------------------------------------

\end{document}
